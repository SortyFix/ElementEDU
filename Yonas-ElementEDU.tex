\documentclass{article}
\usepackage{graphicx} % Required for inserting images

\title{ElementEDU - Verlauf}
\author{Yonas Nieder Fernández}
\date{Oktober 2023}

\begin{document}

\maketitle

\section{Einführung}

Wir, Falk Odin Ivo Quiring und Yonas Nieder Fernández, haben es uns mit der Besonderen Lernleistung (BLL) in Informatik zur Aufgabe gesetzt, eine Managementsoftware für Schulen zu entwickeln, welche grundlegende Funktionen für das Hochladen und Einsehen von Hausaufgaben, die Kommunikation zwischen Nutzern, die Einsendung und Überprüfung von Krankmeldungen und einfache Personalisierungsmöglichkeiten für Nutzer bereitstellt.

Das Programm wird in Java und TypeScript geschrieben, wobei wir mehrere Pakete von Drittanbietern, insbesondere Spring Boot und Angular, zur Erweiterung der Entwicklungsmöglichkeiten in unser Projekt integrieren (siehe Abschnitt "Pakete"). 

Unser Projekt, welches wir "ElementEDU" nennen, soll DSGVO-konform sein und mithilfe von Antiviren-Unterstützung das System und andere Nutzer vor Schadsoftware bewahren. Das Design wird klar separiert, um nicht nur benutzerfreundlich zu sein, sondern vor allem die Barrierefreiheit in unserer Software zu fördern. 

\section{Pakete}
\textbf{Java:}
\begin{itemize}
\item Spring Boot
\item Hibernate
\item JPA
\item Docker
\item JJWT
\item Jetbrains Annotations
\item Lombok
\item ClamAV
\item zxing-core
\item commons-codec
\item MariaDB
\item H2
\end{itemize}
\textbf{TypeScript:}
\begin{itemize}
    \item Angular
    \item JWT Decode
    \item rxjs
    \item tslib


\end{itemize}
\end{document}
